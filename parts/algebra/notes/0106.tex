\documentclass{article}
\usepackage{amsmath,amssymb,amsthm,textcomp,gensymb}
\usepackage{mathtools}
\renewcommand{\qedsymbol}{$\blacksquare$}

\setlength{\topmargin}{0.5in}
\usepackage[margin=4cm]{geometry}
\usepackage{enumerate}

\usepackage{setspace}
\onehalfspacing
\usepackage{parskip}
\setlength{\parskip}{0.5em}
\usepackage[T1]{fontenc}
\usepackage{palatino}

% useful characters/operators
\newcommand{\R}{\mathbb{R}}
\newcommand{\C}{\mathbb{C}}
\newcommand{\Z}{\mathbb{Z}}
\newcommand{\Q}{\mathbb{Q}}
\newcommand{\N}{\mathbb{N}}
\newcommand{\matP}{\mathbb{P}}
\newcommand{\matS}{\mathbb{S}}
\newcommand{\matH}{\mathbb{H}}
\newcommand{\matT}{\mathbb{T}}
\newcommand{\st}{\ s.t.\ }
\newcommand{\ie}{\ i.e.\ }
\newcommand{\eg}{\ e.g.\ }
\def \diam {\operatorname{diam}}
\def \Hom {\operatorname{Hom}}
\def \id {\operatorname{id}}
\def \tr {\operatorname{tr}}
\def \dist {\operatorname{dist}}
\def \intr {\operatorname{int}}
\def \sgn {\operatorname{sgn}}
\def \im {\operatorname{Im}}
\def \re {\operatorname{Re}}
\def \curl {\operatorname{curl}}
\def \divg {\operatorname{div}}
\def \GL {\operatorname{GL}}
\def \End {\operatorname{End}}
\def \Aut {\operatorname{Aut}}
\newcommand{\pdr}[2]{\dfrac{\partial #1}{\partial #2}}
\newcommand{\dr}[2]{\dfrac{\text{d} #1}{\text{d} #2}}
\newcommand{\df}{\text{d}}
\newcommand{\inner}[2]{\left\langle #1, #2\right\rangle}

% arrows and :=, =:
\makeatletter
\providecommand*{\twoheadrightarrowfill@}{%
  \arrowfill@\relbar\relbar\twoheadrightarrow
}
\providecommand*{\twoheadleftarrowfill@}{%
  \arrowfill@\twoheadleftarrow\relbar\relbar
}
\providecommand*{\xtwoheadrightarrow}[2][]{%
  \ext@arrow 0579\twoheadrightarrowfill@{#1}{#2}%
}
\providecommand*{\xtwoheadleftarrow}[2][]{%
  \ext@arrow 5097\twoheadleftarrowfill@{#1}{#2}%
}
\makeatother

\newcommand{\defeq}{\vcentcolon=}
\newcommand{\eqdef}{=\mathrel{\mathop:}}

% integral for measure theory
\newcommand{\lowerint}{\underline{\int_{\R^d}}}
\newcommand{\upperint}{\overline{\int_{\R^d}}}
\newcommand{\lint}[1]{\underline{\int_{\R^d}} #1 (x)dx}
\newcommand{\uint}[1]{\overline{\int_{\R^d}} #1 (x)dx}
\newcommand{\sint}[1]{\simp{\int_{\R^d} #1 (x)dx}}
\newcommand{\lesint}[1]{\int_{\R^d} #1 (x)dx}

% note taking
\newcommand{\fancyem}[1]{\underline{\textsc{#1}}}

% theorem style
\newtheorem*{theorem}{Theorem}
\newtheorem*{corollary}{Corollary}
\newtheorem*{lemma}{Lemma}
\newtheorem*{conjecture}{Conjecture}
\newtheorem*{proposition}{Proposition}

\theoremstyle{definition}
\newtheorem*{definition}{Definition}
\newtheorem*{example}{Example}
\theoremstyle{remark}
\newtheorem*{remark}{Remark}

% for clearer reference
\usepackage{hyperref}
\newcommand{\corollaryautorefname}{Corollary}
\newcommand{\lemmaautorefname}{Lemma}
\newcommand{\definitionautorefname}{Definition}
\newcommand{\exampleautorefname}{Example}
\newcommand{\conjectureautorefname}{Conjecture}
\renewcommand{\subsectionautorefname}{Section}

% other styling
\usepackage{fancyvrb, fancyhdr}
\usepackage{tikz}
\usepackage{tcolorbox}

\usepackage{tikz-cd}

\begin{document}
\renewcommand{\ref}[1]{\autoref{#1}}
\title{Math 494}
\author{Yiwei Fu}
\date{Jan 6, 2022}
\maketitle

\section*{Introduction and Administration}
Two highlights of the course:
\begin{enumerate}
  \item
  For a quadratic equation $ax^2 + bx + c = 0 (a \neq 0)$ there are roots
  \[
  x = \frac{-b \pm \sqrt{b^2 - 4ac}}{2a}.
  \]
  There is similar formulas for polynomials of degree $3$ and $4$, but not degree $5$ or higher.

  \item
  $x^2 + 3 = y^3$ has only integer solutions $x = \pm 5, y = 3$. 
  Important and powerful methods behind simple problems.
\end{enumerate}

\begin{itemize}
  \item HW due Monday midnight
  \item Office hours: Sunday 2 - 3:30 p.m., Friday 7 - 8:30 p.m.
\end{itemize}

\section*{Rings}
\subsection*{Review}
\begin{definition}
  A group $(G, *)$ is a set $G$ equipped with function $*:G \times G \to G$ and element $1_G \in G \st$
  \begin{itemize}
    \item $*$ is associative
    \item $1_G$ is the identity element (hence unique)
    \item $\forall g \in G, \exists h \in G \st gh = hg = 1_G.$
  \end{itemize}
\end{definition}
Note that the identity and inverse elements are unique.

We say $G$ is abelian if $ab = ba, \forall a, b \in G$.

\begin{definition}
  Group homomorphism
\end{definition}

\begin{lemma}
  If $G,H$ are groups and $H$ is abelian, then $\Hom(G,H)$ is an abelian group under the operation
  \[
  \varphi + \psi: G \to H, g \mapsto \varphi(g) + \psi(g).  
  \]
\end{lemma}

\begin{definition}
A \underline{ring} is a set $R$ with two functions $+, \cdot: R \times R \to R \st$:
\begin{itemize}
\item $(R,+)$ is an abelian group (identity $0 \in R$, inverse element $-r$ for every $r \in R$).
\item $(R, \cdot)$ is associative with an identity $1.$
\item Distribution laws hold
\[
a\cdot(b+c) = a\cdot b + a \cdot c,\quad (b+c) \cdot a = b \cdot a + c \cdot a.
\]
\end{itemize}
\end{definition}

Hence if $G$ is an abelian group, then $\End(G)$ is a ring.

\begin{definition}
Let $R, S$ be rings. A \underline{ring homomorphism} is a function $f: R \to S$ which preserves $+, \cdot, 0, 1,$ additive inverses. (In fact it suffices to preserve $+, \cdot, 1$ since $f$ preserves $+$ implies it preserves $0$ and inverses.)
\begin{align*}
f(r +_R r') & = f(r) +_S f(r') \\
f(r \cdot_R r') & = f(r) \cdot_S f(r') \\
f(1_R) & = 1_S.
\end{align*}
\end{definition}


\end{document}