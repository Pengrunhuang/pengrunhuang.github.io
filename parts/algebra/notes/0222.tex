\documentclass{article}
\usepackage{amsmath,amssymb,amsthm,textcomp,gensymb,nccmath}
\usepackage{mathtools}
\renewcommand{\qedsymbol}{$\blacksquare$}

\setlength{\topmargin}{0.5in}
\usepackage[margin=4cm]{geometry}
\usepackage{enumerate}

\usepackage{setspace}
\onehalfspacing
\usepackage{parskip}
\setlength{\parskip}{0.5em}
\usepackage[T1]{fontenc}
\usepackage{palatino}

% useful characters/operators
\newcommand{\R}{\mathbb{R}}
\newcommand{\C}{\mathbb{C}}
\newcommand{\Z}{\mathbb{Z}}
\newcommand{\Q}{\mathbb{Q}}
\newcommand{\N}{\mathbb{N}}
\newcommand{\matP}{\mathbb{P}}
\newcommand{\matS}{\mathbb{S}}
\newcommand{\matH}{\mathbb{H}}
\newcommand{\matT}{\mathbb{T}}
\newcommand{\st}{\ s.t.\ }
\newcommand{\ie}{\ i.e.\ }
\newcommand{\eg}{\ e.g.\ }
\def \diam {\operatorname{diam}}
\def \Hom {\operatorname{Hom}}
\def \id {\operatorname{id}}
\def \tr {\operatorname{tr}}
\def \dist {\operatorname{dist}}
\def \intr {\operatorname{int}}
\def \sgn {\operatorname{sgn}}
\def \im {\operatorname{Im}}
\def \re {\operatorname{Re}}
\def \curl {\operatorname{curl}}
\def \divg {\operatorname{div}}
\def \GL {\operatorname{GL}}
\def \End {\operatorname{End}}
\def \Aut {\operatorname{Aut}}
\def \Fr {\operatorname{Frac}}
\def \cont {\operatorname{cont}}
\newcommand{\pdr}[2]{\dfrac{\partial #1}{\partial #2}}
\newcommand{\dr}[2]{\dfrac{\text{d} #1}{\text{d} #2}}
\newcommand{\df}{\text{d}}
\newcommand{\inner}[2]{\left\langle #1, #2\right\rangle}

% arrows and :=, =:
\makeatletter
\providecommand*{\twoheadrightarrowfill@}{%
  \arrowfill@\relbar\relbar\twoheadrightarrow
}
\providecommand*{\twoheadleftarrowfill@}{%
  \arrowfill@\twoheadleftarrow\relbar\relbar
}
\providecommand*{\xtwoheadrightarrow}[2][]{%
  \ext@arrow 0579\twoheadrightarrowfill@{#1}{#2}%
}
\providecommand*{\xtwoheadleftarrow}[2][]{%
  \ext@arrow 5097\twoheadleftarrowfill@{#1}{#2}%
}
\makeatother
\newcommand{\defeq}{\vcentcolon=}
\newcommand{\eqdef}{=\mathrel{\mathop:}}

% integral for measure theory
\newcommand{\lowerint}{\underline{\int_{\R^d}}}
\newcommand{\upperint}{\overline{\int_{\R^d}}}
\newcommand{\lint}[1]{\underline{\int_{\R^d}} #1 (x)dx}
\newcommand{\uint}[1]{\overline{\int_{\R^d}} #1 (x)dx}
\newcommand{\sint}[1]{\simp{\int_{\R^d} #1 (x)dx}}
\newcommand{\lesint}[1]{\int_{\R^d} #1 (x)dx}

% note taking
\newcommand{\fancyem}[1]{\underline{\textsc{#1}}}

% theorem style
\newtheorem*{theorem}{Theorem}
\newtheorem*{corollary}{Corollary}
\newtheorem*{lemma}{Lemma}
\newtheorem*{conjecture}{Conjecture}
\newtheorem*{proposition}{Proposition}

\theoremstyle{definition}
\newtheorem*{definition}{Definition}
\newtheorem*{example}{Example}
\theoremstyle{remark}
\newtheorem*{remark}{Remark}

% for clearer reference
\usepackage{hyperref}
\newcommand{\corollaryautorefname}{Corollary}
\newcommand{\lemmaautorefname}{Lemma}
\newcommand{\definitionautorefname}{Definition}
\newcommand{\exampleautorefname}{Example}
\newcommand{\conjectureautorefname}{Conjecture}
\renewcommand{\subsectionautorefname}{Section}

% other styling
\usepackage{fancyvrb, fancyhdr}
\usepackage{tikz}
\usepackage{tcolorbox}

\pagestyle{fancy}
\fancyhead[LO,L]{Feb 22, 2022}
\fancyhead[RO,R]{Yiwei Fu}
\fancyhead[C]{MATH 494}
\fancyfoot[CO,C]{\thepage}

\usepackage{tikz-cd}

\usepackage{epigraph}

% \epigraphsize{\small}% Default
\setlength\epigraphwidth{8cm}
\setlength\epigraphrule{0pt}

\usepackage{etoolbox}

\makeatletter
\patchcmd{\epigraph}{\@epitext{#1}}{\itshape\@epitext{#1}}{}{}
\makeatother


\begin{document}
% \renewcommand{\ref}[1]{\autoref{#1}}
% \title{Math 494}
% \author{Yiwei Fu}
% \date{Feb 03, 2022}
% \maketitle

\epigraph{Artin's books is a little short on fields.}{-- \textup{Michiael Zieve}}

Today: $\sqrt[3]{2} \neq \sqrt{a_1} + \sqrt{a_2} + \ldots + \sqrt{a_k}, a_i \in \Q, a_i > 0$.

We show that the left-hand side is in a field $K \st K$ has dimension divisible by $3$ or $\infty$ as $\Q$ vector space, while the right-hand side has dimension $2^\ell$.

\begin{definition}
  Given a field $K$, a field $L$ containing $K$ is called an \emph{extension} of $K$, and $L/_K$ (not a quotient) is a "field extension."
\end{definition}

\begin{definition}
  If $L/_K$ is a field extension then its \emph{degree} is $\dim_KL \defeq \dim L$ as $K$ vector space.
\end{definition}

Let $L/_K$ be a field extension and let $\alpha \in L$. Then $K(\alpha)$ denotes the smallest field that contains $K$ and $\alpha = \left\lbrace \frac{a(\alpha)}{b(\alpha)}: a, b \in K[x], b(\alpha) \neq 0\right\rbrace$.

Let $S = \{f(x) \in K[x]: f(\alpha) = 0\}$. Then $S$ is an ideal of $K[x]$. Since $K[x]$ is PID, $S = (m(x))$ for some $m(x) \in K[x]$. We may assume $m(x)$ is monic or $0$. (Clearly $m$ is irreducible.)

\begin{definition}
  $\alpha$ is \emph{algebraic} over $K$ if $f(\alpha) = 0$ for some nonzero $f(\alpha) \in K[x]$. In this case, the \emph{minimal polynomial} of $\alpha$ over $k$ is $\operatorname{minpol}_K(\alpha) = \operatorname{irr}_K(\alpha) =$ the unique monic irreducible $m(x) \in K[x] \st m(\alpha) = 0$.

  $\alpha$ is transcendental over $K$ is $\alpha$ is not algebraic over $K$.
\end{definition}

If $\alpha$ is transcendental over $K$ then $K[x] \xhookrightarrow[]{\text{eval}} K[\alpha] \subseteq K(\alpha)$ this homomorphism extends to an injective homomorphism $K(x) \hookrightarrow K(\alpha)$ which is an isomorphism $K(x) \cong K(\alpha)$.

Now suppose $\alpha$ is algebraic over $K$ with $m(x) = \operatorname{minpol}_K(\alpha)$. Then $K[x] \to K[\alpha]$ is surjective with kernel $(m(x))$. So $K[x]/(m(x)) \cong K[\alpha]$. But $(m(x))$ is maximal $\implies K[\alpha]$ is a field $\implies K(\alpha) = K[\alpha]$.

$K[x]/m(x)K[x]$ as a $K$-vector space has basis $\{1, x, x^2, \ldots, x^{n-1}\}$ where $n \defeq \deg(m(x))$ (since every coset $K[x]/(m(x))$ contains a unique polynomial of degree $< n$.)

So if $\alpha$ is algebraic over $K$, and $n \defeq \deg(\operatorname{minpol}_K(\alpha))$, then $K(\alpha)$ has basis $1, \alpha, \ldots, \alpha^{n-1}$ as a $K$-vector space $\implies \dim_K(K(\alpha)) = n = \deg(\operatorname{minpol}_K(\alpha))$.

Notation: if $L/_K$ is a field extension, then $[L:K] = \dim_K L = \dim L$ as a $K$ vector space.

\begin{example}
  $[\Q(i): \Q] = 2$. 
  
  $[\Q(\sqrt[n]{2}): \Q] = n$. (since $x^n - 2$ is irreducible)

  $[\Q(e^{2\pi i/p}): \Q] = p - 1$ if $p$ is prime since $\operatorname{minpol}_\Q(e^{2\pi i/p}) = x^{p-1} + x^{p-2} + \ldots + 1$.
\end{example}

\begin{example}
  Any $3$-dimensional $\C$-vector space has dimension $6$ as an $\R$-vector space: if $\alpha_1, \alpha_2, \alpha_3$ is a $\C$-basis, then every element of the vector space can be written in exactly one way as $(a_1 + ib_1)\alpha_1 + (a_2 + ib_2)\alpha_2 + (a_3 + ib_3)\alpha_3$ with $a_j, b_j \in \R$ i.e. as $a_1\alpha_1 + ib_1\alpha_1 + a_2\alpha_2 + ib_2\alpha_2 + a_3\alpha_3 + ib_3\alpha_3$. So $\alpha_j, i\alpha_j$'s is an $\R$-basis of $V$.
\end{example}

\begin{proposition}
  More generally, if $L/_K$ is a field extension and $V$ is an $L$-vector space of nonzero dimension. Then $\dim_K V = [L: K] \dim_L V$.
\end{proposition}
\begin{proof}
  Let $\alpha_1, \alpha_2, \ldots$ be an $L$-basis for $V$. Let $\beta_1, \beta_2, \ldots$ be a $K$-basis for $L$. We'll show that $\{\alpha_i\beta_j\}$ is a $K$-basis for $V$.

  Every element of $V$ can be written in exactly one way as $\sum_i \ell_i\alpha_i$ with $\ell_i \in L$ and each $\ell_i$ can be written in exactly one way as $\ell_i = \sum_j k_{ij} beta_j$ with $k_{ij} \in K$.

  Every element of $V$ can be written in exactly one way as $\sum_i\sum_j k_{ij} \beta_j \alpha_i \implies \{\alpha_i\beta_j\}$ is a $K$-basis for $V$.
\end{proof}

\begin{corollary}
  If $K, L, M$ are fields with $M \supset L \supset K$ then $[M: K] = [M: L]\cdot[L:K]$.
\end{corollary}

\begin{example}
  Eisenstein $\implies x^3 - 2$ irreducible over $\Q \implies [\Q(\sqrt[3]{2}): \Q] = 3 \implies$ if $K$ is a field containing $\sqrt[3]{2}$ then $[K: Q]$ is divisible by $3$ (or $\infty$.)

  If $L/_K$ is an extension and $\alpha \in K$ has a square root in $L$, then $[K(\sqrt{a}): K] = 1$ or $2$.

  So if $a_1, \ldots, a_n \in \Q$ then $[\Q(\sqrt{a_1}): \Q] = 1$ or $2$, $[\Q(\sqrt{a_1}, \sqrt{a_2}): \Q(\sqrt{a_1})] = 1$ or $2$, $\ldots$. Hence if $K_i \defeq \Q(\sqrt{a_1}, \ldots, \sqrt{a_n})$ then $[K_{i+1}: K_i] = [K_i(\sqrt{a_{i+1}}):K_i] = 1$ or $2$.

  This shows that $[K_i: \Q]$ divides $2^i \implies$ not divisible by $3$.

  We can also get that $\sqrt[3]{2} \neq$ sum of nested square roots.
\end{example}

\epigraph{You know that a protractor is? WOW! I didn't know what computer was when I was in school. I thought there must be some trade-off!}{-- \textup{Micheal Zieve}}





\end{document}
