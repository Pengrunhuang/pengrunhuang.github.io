\documentclass{article}
\usepackage{amsmath,amssymb,amsthm,textcomp,gensymb}
\usepackage{mathtools}
\renewcommand{\qedsymbol}{$\blacksquare$}

\setlength{\topmargin}{0.5in}
\usepackage[margin=4cm]{geometry}
\usepackage{enumerate}

\usepackage{setspace}
\onehalfspacing
\usepackage{parskip}
\setlength{\parskip}{0.5em}
\usepackage[T1]{fontenc}
\usepackage{palatino}

% useful characters/operators
\newcommand{\R}{\mathbb{R}}
\newcommand{\C}{\mathbb{C}}
\newcommand{\Z}{\mathbb{Z}}
\newcommand{\Q}{\mathbb{Q}}
\newcommand{\N}{\mathbb{N}}
\newcommand{\matP}{\mathbb{P}}
\newcommand{\matS}{\mathbb{S}}
\newcommand{\matH}{\mathbb{H}}
\newcommand{\matT}{\mathbb{T}}
\newcommand{\st}{\ s.t.\ }
\newcommand{\ie}{\ i.e.\ }
\newcommand{\eg}{\ e.g.\ }
\def \diam {\operatorname{diam}}
\def \Hom {\operatorname{Hom}}
\def \id {\operatorname{id}}
\def \tr {\operatorname{tr}}
\def \dist {\operatorname{dist}}
\def \intr {\operatorname{int}}
\def \sgn {\operatorname{sgn}}
\def \im {\operatorname{Im}}
\def \re {\operatorname{Re}}
\def \curl {\operatorname{curl}}
\def \divg {\operatorname{div}}
\def \GL {\operatorname{GL}}
\def \End {\operatorname{End}}
\def \Aut {\operatorname{Aut}}
\newcommand{\pdr}[2]{\dfrac{\partial #1}{\partial #2}}
\newcommand{\dr}[2]{\dfrac{\text{d} #1}{\text{d} #2}}
\newcommand{\df}{\text{d}}
\newcommand{\inner}[2]{\left\langle #1, #2\right\rangle}

% arrows and :=, =:
\makeatletter
\providecommand*{\twoheadrightarrowfill@}{%
  \arrowfill@\relbar\relbar\twoheadrightarrow
}
\providecommand*{\twoheadleftarrowfill@}{%
  \arrowfill@\twoheadleftarrow\relbar\relbar
}
\providecommand*{\xtwoheadrightarrow}[2][]{%
  \ext@arrow 0579\twoheadrightarrowfill@{#1}{#2}%
}
\providecommand*{\xtwoheadleftarrow}[2][]{%
  \ext@arrow 5097\twoheadleftarrowfill@{#1}{#2}%
}
\makeatother

\newcommand{\defeq}{\vcentcolon=}
\newcommand{\eqdef}{=\mathrel{\mathop:}}

% integral for measure theory
\newcommand{\lowerint}{\underline{\int_{\R^d}}}
\newcommand{\upperint}{\overline{\int_{\R^d}}}
\newcommand{\lint}[1]{\underline{\int_{\R^d}} #1 (x)dx}
\newcommand{\uint}[1]{\overline{\int_{\R^d}} #1 (x)dx}
\newcommand{\sint}[1]{\simp{\int_{\R^d} #1 (x)dx}}
\newcommand{\lesint}[1]{\int_{\R^d} #1 (x)dx}

% note taking
\newcommand{\fancyem}[1]{\underline{\textsc{#1}}}

% theorem style
\newtheorem*{theorem}{Theorem}
\newtheorem*{corollary}{Corollary}
\newtheorem*{lemma}{Lemma}
\newtheorem*{conjecture}{Conjecture}
\newtheorem*{proposition}{Proposition}

\theoremstyle{definition}
\newtheorem*{definition}{Definition}
\newtheorem*{example}{Example}
\theoremstyle{remark}
\newtheorem*{remark}{Remark}

% for clearer reference
\usepackage{hyperref}
\newcommand{\corollaryautorefname}{Corollary}
\newcommand{\lemmaautorefname}{Lemma}
\newcommand{\definitionautorefname}{Definition}
\newcommand{\exampleautorefname}{Example}
\newcommand{\conjectureautorefname}{Conjecture}
\renewcommand{\subsectionautorefname}{Section}

% other styling
\usepackage{fancyvrb, fancyhdr}
\usepackage{tikz}
\usepackage{tcolorbox}

\usepackage{tikz-cd}

\begin{document}
\renewcommand{\ref}[1]{\autoref{#1}}
\title{Math 494}
\author{Yiwei Fu}
\date{Jan 20, 2022}
\maketitle

\fancyem{Last time} $R =$ ring. $f, g \in R[x], g \neq 0$. If the leading coefficient of $g$ is a unit in $R$ then $\exists q, r \in R[x] \st f = gq + r$ and $\deg(r) < \deg(g).$
\begin{corollary}
For $\alpha \in R$ and $f(x) \in R[x]$, $\exists q(x) \in R[x] \st f(x) = (x - \alpha)q(x) + c, c \in R$. Evaluate this at $\alpha \implies c = f(\alpha)$.
\end{corollary}

\begin{example}
In $\Z[x], 4x^3 + x = (2x)2x^2 + x$. But $4x^3 + x \neq (2x)q(x) + r(x)$ with $\deg(r) < \deg(2x).$ 
\end{example}

If $K$ is a field, what are the ideals in $K[x]$? 

\fancyem{Answer} Any nonzero ideal in $K[x]$ is $(g(x))$ where $g(x)$ is any nonzero element of $I$ having the smallest possible degree.
\begin{proof}
For $f(x) \in I$, $f = gq + r, q,r \in K[x], \deg(r) < \deg(g)$. Bur $r = f - gq \in I$, so the minimality of $\deg(g) \implies r = 0 \implies g \mid f, \ie f \in (g)$.  
\end{proof}

\begin{definition}
In a ring $R$, for any $\alpha \in R$, $(\alpha) \defeq \alpha R$ is called a "principal ideal".
\end{definition}

\fancyem{Note} $(\alpha) = (\alpha u)$ for any $u \in R^*$. If $R =$ integral domain, $\alpha, \beta \in R$, then $(\alpha) = (\beta) \iff \alpha = \beta u, u \in R^*$.
\begin{proof}
$\alpha = \beta x, \beta = \alpha y,  x, y \in R$. Then $\alpha = \beta x = (\alpha y) x \implies \alpha(1 - yx) = 0.$

Then $\alpha = \beta = 0$ or $yx = 1 \implies x, y \in R^*$.
\end{proof}

Units in $R[x]$:

If $R =$ integral domain, $(R[x])^* = R^*$.

If $R = \Z/4\Z, (R[x])^* = 1+2R[x]$. For $y \in R[x], (1 + 2y)^2 = 1 + 4y + 4y^2 = 1.$ If $f, g \in R[x]$ satisfy $fg = 1$ then apply homomorphism: $\varphi: R[x] \to (R/(2))[x]$ to get $\varphi(f) \cdot \varphi(g) = 1$ in $(R/(2))[x] = (\Z/2\Z)[x] \implies \varphi(f) = \varphi(g) = 1 \implies f, g = 1 \pmod{2} \implies f = 1 + 2A, g = 1 + 2B, A, B \in R[x] \implies fg = 1 + 2(A + B) + 4AB = 1 + 2(A + B)$ which is $1$ iff $A = B + 2C \implies f = 1 + 2(B + 2C) = 1 + 2B = g.$

$(\Z/(6)[x])^* = \{\pm 1\}$.

$R, S$ rings, $R \times S$ with coordinate wise addiction and multiplication is a "production ring".

\underline{Ring extensions}

Some examples:
\[
\Z[i] = \Z[x]/(x^2 + 1), \Z\left[\frac{1}{2}\right] = \Z[x]/(2x - 1)
\]
If $R$ is a ring and $I$ is an ideal of $R[x]$, then $R[x]/I$ is a ring and $\exists f: R \to R[x]/I, r \mapsto r(x) \mapsto r + I$ is a homomorphism but not necessarily injective. 

$(\Z/(4)[x])/(2x - 1).$ Let $u =$ image of $x$ in this ring. Then $2u = 1 \implies 0 = 4u^2 = 1 \implies $ it is a zero ring.

If $f(x)$ is a monic polynomial in $R[x]$ of degree $n$ and $S \defeq R[x]/(f(x))$ then each element of $S$ can be written in exactly one way as $a(x) + (f(x))$ with $\deg(a) < n.$

Since if $g(x) \in R[x]$ then $g = fq + r$ for some unique $q, r \in R[x] \st \deg(r) < n$.

$K =$ field: $K[x]$ is a principal idea domain, so: for $f, g \in K[x]$, the ideal $(f, g) = (h)$ for some $h \in K[x].$

So:
\begin{align*}
h & = uf + vg, u, v \in K[x] \\
f & = hr, g = hs, r, s \in K[x].
\end{align*}

And if $w \in K[x]$ divides both $f$ and $g$ then $w \mid h$.

If $p(x) \in K[x]$ is irreducible (non-zero, non-unit, and not a product of two non-units) and $p \mid fg$, then $p \mid f$ or $p \mid g$.
\begin{proof}
If $p \nmid f$ then $(p, f) = (h)$ where $h \mid p$ and $h \mid f$.

Since $p$ is irreducible, either $h = p \cdot$unit or $h =$ unit. Since $p \nmid f$, $h \neq p \cdot$unit. Hence $h = $ unit and since $(h) = 1$ we have $h = 1$.

Hence $pu + fv = 1$. Multiply both sides by $g$ we have $pug + fvg = g$. By hypothesis $fg$ is divisible by $p$. Hence $p \mid g$.
\end{proof}


\end{document}