\documentclass{article}
\usepackage{amsmath,amssymb,amsthm,textcomp,gensymb,nccmath}
\usepackage{mathtools}
\renewcommand{\qedsymbol}{$\blacksquare$}

\setlength{\topmargin}{0.5in}
\usepackage[margin=4cm]{geometry}
\usepackage{enumerate}

\usepackage{setspace}
\onehalfspacing
\usepackage{parskip}
\setlength{\parskip}{0.5em}
\usepackage[T1]{fontenc}
\usepackage{palatino}

% useful characters/operators
\newcommand{\R}{\mathbb{R}}
\newcommand{\C}{\mathbb{C}}
\newcommand{\Z}{\mathbb{Z}}
\newcommand{\Q}{\mathbb{Q}}
\newcommand{\N}{\mathbb{N}}
\newcommand{\matP}{\mathbb{P}}
\newcommand{\matS}{\mathbb{S}}
\newcommand{\matH}{\mathbb{H}}
\newcommand{\matT}{\mathbb{T}}
\newcommand{\st}{\ s.t.\ }
\newcommand{\ie}{\ i.e.\ }
\newcommand{\eg}{\ e.g.\ }
\def \diam {\operatorname{diam}}
\def \Hom {\operatorname{Hom}}
\def \id {\operatorname{id}}
\def \tr {\operatorname{tr}}
\def \dist {\operatorname{dist}}
\def \intr {\operatorname{int}}
\def \sgn {\operatorname{sgn}}
\def \im {\operatorname{Im}}
\def \re {\operatorname{Re}}
\def \curl {\operatorname{curl}}
\def \divg {\operatorname{div}}
\def \GL {\operatorname{GL}}
\def \End {\operatorname{End}}
\def \Aut {\operatorname{Aut}}
\newcommand{\pdr}[2]{\dfrac{\partial #1}{\partial #2}}
\newcommand{\dr}[2]{\dfrac{\text{d} #1}{\text{d} #2}}
\newcommand{\df}{\text{d}}
\newcommand{\inner}[2]{\left\langle #1, #2\right\rangle}

% arrows and :=, =:
\makeatletter
\providecommand*{\twoheadrightarrowfill@}{%
  \arrowfill@\relbar\relbar\twoheadrightarrow
}
\providecommand*{\twoheadleftarrowfill@}{%
  \arrowfill@\twoheadleftarrow\relbar\relbar
}
\providecommand*{\xtwoheadrightarrow}[2][]{%
  \ext@arrow 0579\twoheadrightarrowfill@{#1}{#2}%
}
\providecommand*{\xtwoheadleftarrow}[2][]{%
  \ext@arrow 5097\twoheadleftarrowfill@{#1}{#2}%
}
\makeatother

\newcommand{\defeq}{\vcentcolon=}
\newcommand{\eqdef}{=\mathrel{\mathop:}}

% integral for measure theory
\newcommand{\lowerint}{\underline{\int_{\R^d}}}
\newcommand{\upperint}{\overline{\int_{\R^d}}}
\newcommand{\lint}[1]{\underline{\int_{\R^d}} #1 (x)dx}
\newcommand{\uint}[1]{\overline{\int_{\R^d}} #1 (x)dx}
\newcommand{\sint}[1]{\simp{\int_{\R^d} #1 (x)dx}}
\newcommand{\lesint}[1]{\int_{\R^d} #1 (x)dx}

% note taking
\newcommand{\fancyem}[1]{\underline{\textsc{#1}}}

% theorem style
\newtheorem*{theorem}{Theorem}
\newtheorem*{corollary}{Corollary}
\newtheorem*{lemma}{Lemma}
\newtheorem*{conjecture}{Conjecture}
\newtheorem*{proposition}{Proposition}

\theoremstyle{definition}
\newtheorem*{definition}{Definition}
\newtheorem*{example}{Example}
\theoremstyle{remark}
\newtheorem*{remark}{Remark}

% for clearer reference
\usepackage{hyperref}
\newcommand{\corollaryautorefname}{Corollary}
\newcommand{\lemmaautorefname}{Lemma}
\newcommand{\definitionautorefname}{Definition}
\newcommand{\exampleautorefname}{Example}
\newcommand{\conjectureautorefname}{Conjecture}
\renewcommand{\subsectionautorefname}{Section}

% other styling
\usepackage{fancyvrb, fancyhdr}
\usepackage{tikz}
\usepackage{tcolorbox}

\usepackage{tikz-cd}

\begin{document}
% \renewcommand{\ref}[1]{\autoref{#1}}
\title{Math 494}
\author{Yiwei Fu}
\date{Jan 27, 2022}
\maketitle

Finish the proof on Hilbert's Nullstellenstaz.
\begin{corollary}
    If $I$ is an ideal of $\C[x_1, \ldots, x_n]$ generated by $f_1, \ldots, f_k$, and $V$ is the set of all $\alpha \defeq (\alpha_1, \ldots, \alpha_n) \in \C^n \st f_i(\alpha) = 0\ \forall i$, then the maximal ideals of $R/I$ are in bijection with $V$.
\end{corollary}
$R/I$ is called the "coordinate ring".
\begin{proof}
    Correpondence Theorem $\implies$ maximal ideals of $R/I$ are $\pi(M)$ where $\pi: R \twoheadrightarrow R/I$ and $M$ is a maximal ideal of $R$ containing $I$. (also $M \neq M' \implies \pi(M) \neq \pi(M')$)

    An ideal $M$ of $R$ contains $I \iff M$ contains $f_i\ \forall i$.

    $M$ is maximal $\iff M = (x_1 - \alpha_1, x_2 - \alpha_2, \ldots, x_n - \alpha_n)$. So $f_i \in M \iff f_i(\alpha) = 0$.

    So the maximal idals of $R$ containing $I$ are $(x_1 - \alpha_1, \ldots, x_n - \alpha_n)$ where $f_i(\alpha) = 0\ \forall i$.
\end{proof}

\begin{lemma}[Zorn's lemma]
    If a partially ordered set $S$ in which every chain has a upper bound, then $S$ has at least one maximal element.
\end{lemma}
\begin{corollary}
    If $R = $ ring and $I \neq (1)$ is an ideal of $R$, then $I$ is contained in a maximal ideal.
\end{corollary}
\begin{proof}
    Let $S = \{\text{ideals containing $I$ which aren't } (1)\}$ partially ordered under containment. If $T$ is a totally ordered subset of $S$ then let $J = \bigcup_{I' \in T} I'$. $J$ is an ideal not containing $(1)$. We can see that $J \in T$ and is an upper bounde of $T$.

    By Zorn's lemma we conclude that $S$ contains a maximal element, which is a maximum ideal containing $I$.
\end{proof}
\begin{corollary}
    If a ring $R$ has no maximal ideals then $R$ is the zero ring.
\end{corollary}
\begin{corollary}
    If $f_1, \ldots, f_k \in R \defeq \C[x_1, \ldots, x_k]$ have no common zeros in $\C^n$, then the ideal $(f_1, \ldots, f_k)$ is $(1) \ie$
    \[
    1 = g_1f_1 + \ldots + g_kf_k, g_1,\quad \ldots, g_k \in \C[x_1, \ldots, x_k]   
    \] 
\end{corollary}
\begin{proof}
    If $(f_1, \ldots, f_k) \neq (1)$ then it is contained in a maximal ideal of $R$ which is $(x_1 - \alpha_1, \ldots, x_n - \alpha_n)$ where $\alpha_1, \ldots, \alpha_n \in \C$ and $f_i(\alpha) = 0\ \forall i \implies f_i$'s have a common zero, a contradiction.
\end{proof}

\begin{theorem}[Bezout's theorem]
    If $f(x, y)$ and $g(x, y)$ are polynomials in $\C[x, y]$ with no (nonconstant) common factor. Then they only have finitely many common zeros.

    In fact \[\# \text{of zeros} \leq (\text{total deg of } f(x, y)) \cdot (\text{total deg of } g(x, y)).\]
\end{theorem}
\begin{proof}
    We have $\C[x, y] = (\C[y])[x] \subseteq (\C(y))[x]$.\

    The ideal $(f, g)$ in $(\C(y))[x]$ is principal, say it's $(h)$ where $h \in (\C(y))[x]$. If $(h) \neq (1)$ then \[h = \frac{h_1(x, y)}{u(y)}, h_1 \in \C[x, y], u \in \C[y], u \neq 0.\]
    But $u(y)$ is a unit in $\C(y)[x] \implies (h) = (h_1)$.

    So we may assume $h \in \C[x, y], (h) \neq (1)$ and $h \mid f, h \mid g$ in $\C(y)[x]$
    \begin{align*}
        & \implies hA = f, hB = g, & A, B \in \C(y)[x] \\
        & \implies hA_1 = fu_1, hB_1 = gu_2, & A_1, B_1 \in \C[x, y], u_1, u_2 \in \C[y]
    \end{align*}

    If $g_1g_2 \in \C^*$ then there is a contradiction. So assume $u_1 \notin \C^*$. Then $u_1$ has a root $\alpha$
    \begin{align*}
        & \implies h(x, \alpha)A_1(x, \alpha) = 0 \text{ in } \C[x] \\
        & \implies h(x, \alpha) = 0 \text{ or } A_1(x, \alpha) \\
        & \implies y - \alpha \mid h(x, \alpha) \text{ or } y - \alpha \mid A_1(x, \alpha) 
    \end{align*}
    The latter is not possible, and we can always factor out $y - \alpha$ from $h(x, \alpha)$ and apply this process until we arrive at a contradiction.
\end{proof}

\end{document}
